\documentclass[a4paper,11pt]{book}
\usepackage[utf8]{inputenc}
\usepackage[italian]{babel}
%\usepackage{times}

\title{iMovies certification authority}
\author{Gottoli Alessandro\\Marchi Nicolò\\Peretti Mattia}

\begin{document}
\maketitle
\tableofcontents

\chapter*{Prefazione}
...

\chapter{La tecnologia utilizzata}
\section{Jsf}
JSF, acronimo di Java Server Faces è il nome della tecnologia utilizzata in tale progetto.

La tecnologia JSF può essere considerata un framework per lo sviluppo di web application basate su Java.

\section{Primefaces}
Le librerie Primefaces costituiscono una serie di componenti grafici utilizzabili all'interno di una web application Jsf.

\section{Perché tale scelta?}
Il framework Jsf nasce con lo scopo di concentrare gli sforzi del programmatore sul backend piuttosto che sul frontend grafico dell'applicazione.

Per tale motivo, in unione con gli ottimi componenti grafici forniti dalle librerie Primefaces, la nostra scelta è ricaduta su questo tipo di tecnologia.

\chapter{Risk management}

\chapter{Sicurezza}

\chapter{Il frontend grafico}

\chapter{Il backend}

\end{document}