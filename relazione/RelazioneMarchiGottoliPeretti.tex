
\documentclass{article}

\usepackage{graphicx}
\usepackage{alltt}
\usepackage{url}
\usepackage{tabularx}
%\usepackage{ngerman}
\usepackage{longtable}
\usepackage[utf8]{inputenc}
\usepackage[italian]{babel}

\newenvironment{prettytablex}[1]{\vspace{0.3cm}\noindent\tabularx{\linewidth}{@{\hspace{\parindent}}#1@{}}}{\endtabularx\vspace{0.3cm}}
\newenvironment{prettytable}{\prettytablex{l X}}{\endprettytablex}



\title{\huge\sffamily\bfseries Descrizione del sistema e Analisi del Rischio}
\author{Nicolò Marchi \and Alessandro Gottoli \and Mattia Peretti}
\date{\today}


\begin{document}
\maketitle

\tableofcontents
\pagebreak


\section{Descrizione del Sistema}

\subsection{Panoramica del sistema}

L'assegnamento per il laboratorio di Sicurezza delle Reti consisteva nell'implementazione di una Certificate Authority riguardante una fittizia compagnia di nome iMovies, che vuole offrire ai suoi clienti dei servizi basati su PKI (Public Key Infrastructure).

---Possibile descrizione PKI

Il sistema iMovies da noi creato si limita a permettere ad utenti già presenti nel database fornito come materiale allegato al libro ``Applied Information Security'' di David Basin,  Patrick Schaller e Micheal Schl\"apfer, la creazione e la firma di certificati che verranno poi usati per la comunicazione sicura tramite e-mail.
\par L'architettura del sistema consiste in una macchina Ubuntu Server 12.04 con installato il server web Tomcat versione 7.
La web application è stata scritta utilizzando il framework Java Server Faces (JSF) per permettere una più facile implementazione visto l'utilizzo del linguaggio Java.
\par Per la gestione del database ci si è affidati al  Relational Database Management System MySql; per la creazione,la firma, la revoca, e tutte le operazioni di gestione dei certificati ci siamo affidati al toolkit OpenSSL, che è un implementazione open-source dei protocolli SSL/TLS. 
\par Per gli archivi di backup dei dati è stato usato il semplice tool tar, presente in ogni distribuzione Unix; per lo scheduling dei backup è stato usato il tool cron, e per il download dei backup ci siamo affidati a   

%This description should provide a high-level
%overview of the system, e.g., suitable for managers, that complements
%the more technical description that follows.


\subsection{Funzionalità del Sistema}
\subsubsection*{Login}
Il sistema offre principalmente due possibilità di login. Una possibilità consiste nel connettersi al portale attraverso un certificato PKCS\#12 riconosciuto dalla Certificate Authority, che permette di bypassare il controllo delle credenziali nel database, visto che si presume che il certificato sia in mano al proprietario dello stesso.
\par La seconda modalità consiste in un canonico form di login deve l'utente deve inserire username e password. Quest'ultima verrà trasformata in un hash SHA-1 e verrà usata nel confronto con gli hash delle password salvati nel database.
\subsubsection*{Modifica informazioni personali}
Il portale offre agli utenti la possibilità di modificare le informazioni personali precedentemente salvate nel database. Si possono modificare tutti i campi, ad eccezione del campo username, che è fisso e funge anche da chiave primaria nelle tuple del database.
\subsubsection*{Rilascio di certificati}
Ad ogni utente viene fornita la possibilità di creare certificati. Alla creazione di un certificato viene generata una chiave privata con crittografia a 4096 bit e crittata con DES3 e una password inserita dall'utente. Dopodiché viene generato e firmato il certificato relativo, con i dati dell'utente salvati nel database.
\subsubsection*{Revoca dei certificati}
Nella sezione di management dei certificati viene fornita la possibilità di revocare selettivamente i certificati dell'utente. Quando un certificato viene revocato, viene generata nuovamente la Certificate Revocation List della Certificate Authority.
\subsubsection*{Download dei certificati}
Viene fornita la possibilità di scaricare i certificati e le relative chiavi private in formato PKCS\#12. Quando si richiede il download del certificato il sistema richiederà all'utente la password usata durante la creazione della chiave privata, e una nuova password che sarà usata per l'esportazione del certificato PKCS\#12. QUest'ultima password dovrà essere inserita quando si importerà il certificato all'interno di un browser.
\subsubsection*{Eliminazione dei certificati}
Quando un utente sceglie di rimuover eun certificato, innanzi tutto quest'ultimo verrà revocato; dopodiché verrà eliminata la chiave privata associata al certificato, e il certificato stesso.
\subsubsection*{Amministrazione del portale}
L'amministratore del portale accede al back-end di amministrazione solamente con un certificato PKCS\#12 già in suo possesso. Nel back-end l'amministratore può vedere quanti e quali certificati sono stati rilasciati, quanti e quali certificati sono stati revocati, e il valore corrente del serial number (che in realtà consiste nel valore del serial number che verrà assegnato al prossimo certificato generato).
\par Inoltre viene effettuato un log di tutti gli accessi al sito, compresi gli accessi effettuati passando attraverso le backdoor.
\subsubsection*{Backup dei dati}
Il sistema esegue un periodico backup di tutte le chiavi private e di tutti i certificati. I backup sono in realtà due, uno totale che viene eseguito ogni settimana il venerdì alle ore 11.00, mentre un backup incrementale che viene eseguito tutte le ore al 40 minuto. In entrambi i casi viene generato un archivio con il comando Unix ``tar''.
\par Vi è poi un server ftp che permette ad un amministratore di scaricare da remoto i backups.

\subsection{Componenti e Sottosistemi}
Come già detto, per l'implementazione del portale è stata usata la tecnologia di Java Server Faces. JSF, acronimo di Java Server Faces può essere considerata un framework per lo sviluppo di web application basate su Java. \'E basato sul design pattern architetturale Model-View-Controller (MVC) ed è descritto da un documento di specifiche (JSR 127) alla cui stesura hanno partecipato aziende quali IBM, Oracle Corporation, Siemens e Sun Microsystems. Il suo scopo è di semplificare lo sviluppo dell'interfaccia utente (UI) di una applicazione Web.
\par A grandi linee il funzionamento del framework JSF si basa su un file di configurazione XML ({\tt faces-config.xml}) in cui vengono definite le viste (sostanzialmente pagine JSP che sfruttano la \emph{taglibrary faces}) e i controllori. Le singole implementazioni sfruttano una servlet di base {\tt FacesServlet} o un filtro il cui mapping è normalmente {\tt /faces/*} o {\tt *.faces}. La {\tt FacesServlet} deve essere registrata nel file XML ({\tt web.xml}) della web application.


Le librerie Primefaces costituiscono una serie di componenti grafici utilizzabili all'interno di una web application Jsf.


Il framework Jsf nasce con lo scopo di concentrare gli sforzi del programmatore sul backend piuttosto che sul frontend grafico dell'applicazione.

Per tale motivo, in unione con gli ottimi componenti grafici forniti dalle librerie Primefaces, la nostra scelta è ricaduta su questo tipo di tecnologia.

List all system components, subdivided, for example, into
  categories such as platforms, applications, data records, etc. For
  each component, state its relevant properties.


\subsection{Interfacce}

Specify  all interfaces and  information flows, from the technical as well as from the
  organizational point of view.

\subsection{Backdoors}

Describe the implemented backdoors. {\bfseries Do not add
    this section to the version of your report that is handed over to
    the team that reviews your system!}

\subsection{Materiale Aggiuntivo}

You may have additional sections according to your needs.


\section{Risk Analysis and Security Measures}

\subsection{Information Assets}

Describe the relevant assets and their required security
  properties. For example, data objects, access restrictions,
  configurations, etc.

\subsection{Threat Sources}

Name and describe potential threat sources.

\subsection{Risks and Countermeasures}

List all potential threats and the
  corresponding countermeasures. Estimate the risk based on 
  the information about the threat, the threat sources and the 
  corresponding countermeasure. For this purpose, use the following three
  tables.

%\subsubsection{Tools}

\begin{center}
\begin{tabular}{|l|l|}
\hline
\multicolumn{2}{|c|}{\bf Impact} \\
\hline
Impact & Description \\
\hline
\hline
High   & \hspace*{20pt}\ldots \\
\hline
Medium & \hspace*{20pt}\ldots \\
\hline
Low   & \hspace*{20pt}\ldots \\
\hline
\end{tabular}
%
%\vspace{5mm}
%
%\noindent \hspace*{10pt}
\begin{tabular}{|l|l|}
\hline
\multicolumn{2}{|c|}{\bf Likelihood} \\
\hline
Likelihood & Description \\
\hline
\hline
High   & \hspace*{20pt}\ldots \\
\hline
Medium & \hspace*{20pt}\ldots \\
\hline
Low   & \hspace*{20pt}\ldots \\
\hline
\end{tabular}
\end{center}

\vspace{5mm}

\begin{center}
\begin{tabular}{|l|c|c|c|}
\hline
\multicolumn{4}{|c|}{{\bf Risk Level}} \\
\hline
{{\bf Likelihood}} & \multicolumn{3}{c|}{{\bf Impact}} \\ %\cline{2-4}
     & Low & Medium & High \\  \hline
 High & Low & Medium & High  \\
\hline
 Medium & Low & Medium & Medium \\
\hline
 Low & Low & Low & Low \\
\hline
\end{tabular}
\end{center}

\subsubsection{{\it Evaluation Asset X}}

Evaluate the likelihood, impact and the resulting risk,  after implementation of the corresponding countermeasures.

\begin{footnotesize}
\begin{prettytablex}{lXp{6.5cm}lll}
No. & Threat & Implemented/planned countermeasure(s) & L & I & Risk \\
\hline
1 & ... & ... & {\it Low} & {\it Low} & {\it Low} \\
\hline
2 & ... & ...& {\it Medium} & {\it High} & {\it Medium} \\
\hline
\end{prettytablex}
\end{footnotesize}



\subsubsection{{\it Evaluation Asset y}}

\begin{footnotesize}
\begin{prettytablex}{lXp{6.5cm}lll}
No. & Threat & Implemented/planned countermeasure(s) & L & I & Risk \\
\hline
1 & ... & ... & {\it Low} & {\it Low} & {\it Low} \\
\hline
2 & ... & ...& {\it Medium} & {\it High} & {\it Medium} \\
\hline
\end{prettytablex}
\end{footnotesize}

\subsubsection{Detailed Description of Selected Countermeasures}

Optionally explain the details of the countermeasures mentioned above.



\subsubsection{Risk Acceptance}

List all medium and high risks, according to the evaluation above. For each risk, propose additional countermeasures that could be implemented to further reduce the risks.

\begin{footnotesize}
\begin{prettytablex}{p{2cm}X}
No. of threat & Proposed countermeasure including expected impact  \\
\hline
... & ... \\
\hline
... & ... \\
\hline
\end{prettytablex}
\end{footnotesize}

\end{document}

%%% Local Variables: 
%%% mode: latex
%%% TeX-master: "../../book"
%%% End: 
